\documentclass[11pt]{article}

\usepackage{mathpazo}    % For Palatino font (incl. math chars.)
\usepackage{setspace}    % For spacing variation such as double space, etc.
\usepackage{natbib}      % For complex bibliography stuff
\usepackage{qtree}       % For trees
\usepackage{linguex}     % For linguistic glosses
%\usepackage{tree-dvips} % For drawing lines/arrows between two points
\usepackage{txfonts}     % For Times font (incl. math chars.)
\usepackage{times}       % For Times font
\usepackage{tipa}        % For Phonetic characters
\usepackage{avm}         % For HPSG-style Attribute Value Matrices
\usepackage{stmaryrd}    % For semantic-y [[...]]
%\usepackage[margin=1.50in]{geometry}

\addtolength{\topmargin}{-0.9in}
\addtolength{\textheight}{1.4in}

\title{ \LaTeX{} for linguists }
\author{Jon Dehdari}
\bibliographystyle{apalike}

\begin{document}
\maketitle
\begin{spacing}{1.2}

\section{Introduction}
\LaTeX{} makes nice looking documents.  By default it does \textsc{Justified Margins}, which looks \emph{really} professional.  It also does \textbf{hyphenation} really well, which is much \underline{harder} to do than it would \textsl{seem}, which is why other programs don't enable hyphenation by default.


\section{Methods}
We'll talk about linguistic glosses, phonetic characters, trees, AVM's, and semantic/logic symbols.


\subsection{Gloss example}
\ex.
I am going to the store.


\exg.
Hem i wokbaot olsem krab \\
He \textsc{Pred} {walk about} {all same} crab \\
`He walks like a crab.'


\subsection{Phonetics example}
We can see in Penguinese \textipa{[k\super w\ae{}:k\textcorner]} ``stale raisin bread'', compared with English \textipa{/steIl "\*reIz1n brEd/}.


\subsection{Tree example}
Doing trees with qtree is just like bracketing sentences.\footnote{Remember to pad each bracket with a space on both sides.}

\ex.
\Tree[.S [.NP [.Det the ] [.N dogs ] ] [.VP [.V eat ] [.NP [.N food ] ] ] ]



\subsection{AVM example}
\begin{avm}
\[
SUBJ & \[ PERS & 3 \\
          NUM  & SG \\
	  \] \\
PRED & \< NP$_{\@1}$, S:\@2 \>\\
...
\]
\end{avm}


\subsection{Semantics/Logic example}
\ex. $ (G \rightarrow E) \lor (G \land \neg C)  $ \hfill 4, UI

$\llbracket$ginocchi$\rrbracket$ is not $\llbracket$gnocchi$\rrbracket$ 

$$\langle \alpha , \beta \rangle \in \Omega \leftrightarrow \forall x \exists y ( x \neq y ) $$


\section*{Results}
Here's a table:\footnote{I actually just made this up.}

\begin{center}

\begin{tabular}{|l|l|l|}
\hline
Red  &  Green  &  Blue \\
\hline
\hline
% I don't know what happened to the other 1%
33\% &  33\%   &  33\% \\
\hline
\end{tabular}

\end{center}


\section{Discussion}
As we have seen in the previous table, $ 33 + 33 + 33 = 100$ for large values of 33.


\end{spacing}
%\bibliography{mybib}
\end{document}
