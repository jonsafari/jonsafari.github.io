\documentclass[11pt]{article}

\usepackage{mathpazo}
\usepackage{setspace}
%\usepackage{natbib}
%\usepackage{tipa}
\usepackage{graphicx}
%\usepackage{amsmath}
\usepackage{amssymb}
%\usepackage[dvipsnames]{color}
%\usepackage{cancel}
%\usepackage[normalem]{ulem}
%\usepackage{qtree}
%\usepackage{linguex}
%\renewcommand{\refdash}{}
%\usepackage{tree-dvips}
%\usepackage{cgloss}
%\usepackage{txfonts}
%\usepackage{times}
\usepackage{units}
%\usepackage{avm}
\usepackage{color}
\definecolor{darkblue}{rgb}{0,0,.45}
\definecolor{blue}{rgb}{0,0,.6}
\definecolor{maroon}{rgb}{.4,0,0}
\definecolor{lightgrey}{rgb}{.7,.7,.7}
\usepackage[pdfborder=0 0 0, colorlinks=true, linkcolor=darkblue, urlcolor={blue}]{hyperref}
\usepackage[vmargin=1.2in, hmargin=1.2in, a4paper]{geometry}
%\addtolength{\topmargin}{-0.75in}
%\addtolength{\textheight}{1.00in}

%\bibliographystyle{lsalikealt}

\newcommand{\bss}{\char`\\\hspace{-.45ex}\char`\\} % Stupid backslash, esp. in code
\newcommand{\bs}{\char`\\} % Stupid backslash, esp. in code
\newcommand{\code}[1]{{\texttt{{\color{lightgrey}\raisebox{.45ex}{$\ulcorner$}\hspace{-.5ex}{\color{maroon}#1}\hspace{-.54ex}\raisebox{.45ex}{$\urcorner$}}}}}

\title{Assignment 1}

\begin{document}
\maketitle
%\noindent%
%{\Large \textsc{Assignment 1} } \\[.55ex]
%{\large \textsc{Linguistics 384} } \\[.25ex]


\begin{spacing}{1.35}
Your boss has given you several files that need to be modified and summarized.
Your elite shell scripting skills will save the day!
Develop a shell script that can achieve the goals listed below, and email Jon the final shell script.
\textbf{Be sure to read and follow all steps}, and include lots of useful comments in the script.\footnote{Start a line with the \, \textbf{\#} \, symbol, followed by your useful comments.}
You can work with one other person in the class on this.

The files are named \texttt{file\_00.txt.gz}, \,\texttt{file\_01.txt.gz}, etc.

\begin{enumerate}
	\item Use \texttt{wget} to download all files from \url{http://languagemodel.org/classes/uds/shell_and_python_basics/hw1_files/}.
		You can use the following command to download the files: \\
		\begin{footnotesize}
		\verb|for i in {00..10}; do| \\
		\verb|  wget -c http://languagemodel.org/classes/uds/shell_and_python_basics/hw1_files/file_$i.txt.gz;| \\
		\verb|done|
		\end{footnotesize} \\
		Explain in a comment line above this command what is happening here.
	\item Use \texttt{gunzip} to uncompress each file.  Use another loop for this step.
	\item Use \texttt{cat} to compile all uncompressed files into a single file.%\footnote{You can use \texttt{zcat} instead if you want.}
	\item Use \texttt{iconv} to convert the text from the old-school encoding \mbox{ISO-8859-1} to modern \mbox{UTF-8}.
	\item Convert all uppercase letters to lowercase.
	\item Use \texttt{grep} to find out how many lines contain ``Twitter''.  There is a command-line argument to grep that will simply tell you how many lines match, rather than print the lines that match.  Also find out how many lines contain ``Google'' and ``Facebook''.
	\item Find out what the top 10 most fequent words are that \emph{start} with ``euro''.  Find out the top 10 words \emph{ending} in ``platz''.
\end{enumerate}

\end{spacing}
%\bibliography{localbib}
\end{document}
